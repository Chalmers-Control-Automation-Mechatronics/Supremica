% -*- TeX-master:"main" -*-

\chapter{Introduction}
\label{introduction}

%Define the problem.
Ever increasing competition forces companies to diversify
their product lines. To achieve customizable products the
companies require constantly changing and evolving
production plants. Such plants are called flexible
production plants. The plant is constantly adopted to the
products with the goal of production of highest quality
products in as little time as posibble. To achieve this the
software for the automatic control of such a plant has to be
developed and upgraded along with the plant. The problem is
that when using today established development methods this
upgrade is impossible without an interrupt of the production
that can last for several weeks in some cases.

To avoid the down time associated with the upgrade of the
control software in the production plants a more agile
architecture for the control software is needed. To be more
specific an object oriented architecutre and development
method are needed as opposed to the loop based architecture
and structured development method implemented with PLCs. The
advantages of object oriented software development are
reaped across the IT industry for quite some time now. Why
it has not been widely adopted in the automatic control
software development is for the most part because of the
high investment already made in the PLCs making the
companies reluctant to invest in yet another expensive
thechnology driven upgrade. In other words, as long as the
automation control problems in the existing plants are
possible to be solved with PLCs there will be no adoption of
object orented architectures. Another big reason for slow
adoption of object oriented development methods is the lack
of standard object oreinted platform suitable for the
automation software applications.

%Define other approaches, state of the art.











%Introduce Function Blocks



%Introduce Automation Objects

