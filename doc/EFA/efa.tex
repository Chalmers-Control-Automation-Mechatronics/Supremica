\documentclass{article}

\usepackage{amsthm,amsmath,amssymb}

\newtheorem{theorem}{Theorem}


\begin{document}
\tableofcontents
\section{Introduction}

\subsection{Background}

\section{Extended Finite Automaton (EFA)}
An Extended Deterministic Finite Automaton (EFA) is an
augmentation of a regular finite automaton. The regular automaton
$A$ is defined as $A=\langle Q,\Sigma,\delta, q_i,Q_m \rangle$.
Let $T=\{(q,\sigma)\in Q \times \Sigma: \delta(q,\sigma)!\}$
denote the set of transitions in the deterministic automaton $A$
and let $\tau_k$, $k=1,...,|T|$ denote the elements (transitions)
of $T$.

The automaton $A$ is extended by a set $X=X_1\times ... \times
X_n$, where $X_k$, $k=1,...,n$, are finite nonempty subsets of
$\mathbb{Z}$. The set $X$ is the domain of definition of an
$n$-tuple of integer variables $x=(x_1,...,x_n)$ with initial
value $x_0=(x_{10},...,x_{n0})$. We also add to $A$ an $|T|$-tuple
of function pairs $G=\big((g_{\tau_1},a_{\tau_1}),...,
(g_{\tau_{|T|}},a_{\tau_{|T|}})\big)$, where each pair
$(g_{\tau_k},a_{\tau_k})$ is associated to the corresponding
transition $\tau_k$ in $T$. $g_{\tau_k}$ is an evaluation of a
boolean formula (a guard formula) over $X$ i.e. $g_{\tau_k}:X
\rightarrow \{0,1\}$ saying if the corresponding transition
$\tau_k$ is enabled or not. $a_{\tau_k}$ is an updating (action)
function from $X$ to $X$ of the integer variables
$x=(x_1,...,x_n)$.

With the above notations we will formally define an EFA as the
$9$-tuple $A_{e}=\langle Q, \Sigma, \delta_{e}, q_i, Q_m, X, G
,x_0, X_m \rangle$, where $X_m$ are the marked integer values of
the variables $x$ and $\delta_{e}: Q \times \Sigma \times X
\rightarrow Q\times X $ is a new extended version of the
transition function $\delta:Q \times \Sigma \rightarrow Q$. The
extended transition function $\delta_e$ is defined by means of
$\delta$ and the elements of $G$ as
\begin{eqnarray}
\delta_{e}(q,\sigma,x)=\left\{
\begin{array}{ll}
(\delta(q,\sigma),a_{q \sigma}(x)\big)\quad\quad g_{q \sigma}(x)=1\\
\textrm{undefined} \quad\quad\quad\quad\quad g_{q \sigma}(x)=0
\end{array}\right.
\end{eqnarray}
If $(q,\sigma)\notin T$ then $g_{q\sigma}$ is undefined and
therefore $\delta_{e}(q,\sigma,x)$ is undefined. Notice that for
the set of transitions $T_e$ in the extended automaton $A_e$ we
have $T_e=\{(q,\sigma)\in Q \times \Sigma: g_{q\sigma}(x)=1
\textrm{ for some } x\in X\}\subset \{(q,\sigma)\in Q \times
\Sigma: g_{q\sigma} \textrm{ is defined } \}\ =T$.



\subsection{Synchronous composition of EFA's}

Formally we define the synchronous product $A||B$ of two EFA's
\begin{eqnarray}
A=\langle Q^A, \Sigma^A, \delta^A, q^A_i, Q^A_m, X^{A},
G^A,x^A_0,X^A_m \rangle
\end{eqnarray} and \begin{eqnarray} B=\langle Q^B, \Sigma^B, \delta^B, q^B_i,
Q^B_m, X^{B}, G^B,x^B_0,X^B_m \rangle
\end{eqnarray} as
\begin{eqnarray}
A||B&=&\langle Q^A\times Q^B, \Sigma^A \cup \Sigma^B ,
\delta^{A||B}, (q^A_i,q^B_i) ,\nonumber\\
&& Q^A_m\times Q^B_m, X^A \times X^B, G^{A||B},(x^A_0,x^B_0),
X^A_m \times X^B_m \rangle.
\end{eqnarray}
To define the guard functions, the action functions and the
transition function of the product we make use of the
corresponding functions in $A$ and $B$. To denote the elements of
$A||B$ we remove unnecessary  superscripts and write $A||B=\langle
Q, \Sigma, \delta, q_i, Q_m, X, G ,x_0, X_m \rangle$. The guard
functions $g_{q \sigma}(x)$ are defined as
\begin{eqnarray}\label{guard_sync_1}
g_{q \sigma}(x)=\left\{
\begin{array}{ll}
g_{q^A\sigma}(x^{A})\wedge g_{q^B\sigma}(x^{B}) \quad\quad\sigma\in \Sigma^A \cap \Sigma^B\\
g_{q^A\sigma}(x^{A})\quad\quad\quad\quad\quad\quad\quad\sigma\in \Sigma^A\setminus\Sigma^B\\
g_{q^B\sigma}(x^{B})\quad\quad\quad\quad\quad\quad\quad\sigma\in\Sigma^B\setminus\Sigma^A
\end{array}\right.
\end{eqnarray}
If any one of the guards functions in the above definition are
undefined the guard function for the product $g_{q \sigma}$ is
also undefined. To simplify the notation further, we use "dummy"
guard functions in the following way. If $\sigma\in \Sigma^A
\setminus \Sigma^B$ we introduce fictional guards functions
$g_{q^B\sigma}(.)=1$ for $B$ and if $\sigma\in\Sigma^B\setminus
\Sigma^A$ we set $g_{q^A\sigma}(.)=1$. Doing this, we can write
equation (\ref{guard_sync_1}) as
\begin{eqnarray}
g_{q \sigma}(x)= g_{q^A\sigma}(x^{A})\wedge g_{q^B\sigma}(x^{B}).
\end{eqnarray}
The updating functions $a_{q \sigma}(x)$ are defined as
\begin{eqnarray}\label{action_sync_1}
a_{q \sigma}(x)=\left\{
\begin{array}{ll}
 \big(a^A_{q^A \sigma}(x^{A}), a^B_{q^B \sigma}(x^{B})\big)\quad\quad\sigma\in \Sigma^A \cap \Sigma^B\\
\big(a^A_{q^A\sigma}(x^{A}),x^{B}\big)\quad\quad\quad\quad\quad\sigma\in\Sigma^A\setminus\Sigma^B\\
\big(x^{A},a^B_{q^B\sigma}(x^{B})\big)\quad\quad\quad\quad\quad\sigma\in\Sigma^B\setminus\Sigma^A
\end{array}\right.
\end{eqnarray}
As in the definition of the guard functions the notion of being
undefined is inherited from the action functions of $A$ and $B$.
Similar to the "dummy" guard functions we introduce fictional
action functions. If $\sigma\in \Sigma^A \setminus \Sigma^B$ we
introduce fictional action functions $a_{q^B\sigma}(x^B)=x^B$ for
$B$ and if $\sigma\in\Sigma^B\setminus \Sigma^A$ we set
$a_{q^A\sigma}(x^A)=x^A$. With this convention equation
(\ref{action_sync_1}) simplifies to
\begin{eqnarray}
a_{q \sigma}(x)=
 \big(a^A_{q^A \sigma}(x^{A}), a^B_{q^B \sigma}(x^{B})\big).
\end{eqnarray}


 The transition function
$\delta(q,\sigma,x)$ is defined from $\delta^A$ and $\delta^B$ as
\begin{eqnarray}
\delta(q,\sigma,x)=\left\{
\begin{array}{ll}
\big(\delta^A(q^A,\sigma),\delta^B(q^B,\sigma),a_{q \sigma}(x)\big) \quad\quad \sigma\in \Sigma^A \cap \Sigma^B \quad g_{q \sigma}(x)=1\\
\big(\delta^A(q^A,\sigma),q^B,a_{q \sigma}(x)\big) \quad\quad\quad\quad\quad \sigma\in \Sigma^A \setminus \Sigma^B \quad g_{q \sigma}(x)=1\\
\big(q^A,\delta^B(q^B,\sigma),a_{q \sigma}(x)\big) \quad\quad\quad\quad\quad \sigma\in \Sigma^B \setminus \Sigma^A \quad g_{q \sigma}(x)=1\\
\textrm{
undefined}\quad\quad\quad\quad\quad\quad\quad\quad\quad\quad\quad\quad\quad\quad\quad\quad\quad
g_{q\sigma}(x)=0
\end{array}\right.
\end{eqnarray}

The synchronous product $A||B$ of two FSA $A$ and $B$ can be
extended to $(A||B)_e$ by taking the synchronous product of the
extended components  $A_e$ and $B_e$. In other words, the
operations extension and product commute, $(A||B)_e=A_e||B_e$.

Moreover, by the above definition of synchronization we now obtain
the following property

\begin{theorem}
   Full synchronous composition of EFA's is an associative
   operation.
\end{theorem}

\noindent \textbf{Proof}: To prove the associativity we
considering three automata $A, A'$ and $A''$, and show that
$(A\|A')\|A'' = A\|(A'\|A'')$. Here, we assume that the three
automata have no common variables.
\newline\newline
\noindent The 8-tuple representing $(A\|A')\|A''$ is
\begin{center}
\begin{tabular}{ll}
  $(A\|A')\|A'' =$ & $\Big \langle (Q \times Q') \times Q'',$\\
  & $ \big(\Sigma \bigcup \Sigma' \big) \bigcup \Sigma'',$\\
  & $\delta^{(A\|A')\|A''},$\\
  & $\big((q_i, q'_i) , q''_i\big),$\\
  & $(Q_m \times Q_m') \times Q_m'',$\\
  & $\big[[X, X'], X'']$\\
  & $G^{(A\|A')\|A''},$\\
  & $\big[[X_m, X'_m], X''_m \big] \Big \rangle$
\end{tabular}
\end{center}

\noindent In order to establish associativity we now have to show
this for each of the elements in the 8-tuple. For the states, the
alphabet, the initial state, the marked states and the variables,
associativity is clear by the laws of set operations. We will now
in turn deal with the guard/action functions and the transition
function.

For the guard functions $g^{(A\|A')\|A''} : X^c \rightarrow
\{\mathrm{true}, \mathrm{false}\}$ of the composite automaton we
have by definition

\begin{equation}
  g^{(A\|A')\|A''}(X^c) = \big[g(X) \wedge g'(X') \big] \wedge g''(X'') = g(X) \wedge \big[g'(X') \wedge g''(X'')] = g^{A\|(A'\|A'')}(X^c)
\end{equation}

\noindent Now consider the action function $a^{(A\|A')\|A''} : X^c
\rightarrow X^c$.
\subsection{The translation of an EFA to  a FA}
To simplify the realization we shall restrict us to quite simple
action- and guard functions. For the updating functions we assume
that $a_{q \sigma}(x)=(a^1_{q \sigma}(x),...,a^n_{q
\sigma}(x))=(a^1_{q \sigma}(x_1),...,a^n_{q \sigma}(x_n))$. In
other words, the updating of the variables are independent from
each other. For the guard functions we assume that they have the
following form $g_{q \sigma}(x)=g^1_{q \sigma}(x_1)\wedge
...\wedge g^n_{q \sigma}(x_n)$. These assumptions makes straight
forward to translate an EFA to a FA.

Given an extended finite automaton $A=\langle Q, \Sigma, \delta,
q_i, Q_m, X, G ,x_0, X_m \rangle$ with simple guard and action
functions, the steps are:\\
- Form a regular FA, $A'$, from $A$ by removing $G$, $X$, $X_m$,
$x_0$ and rename (associate with the corresponding state) all
enabled events $\sigma$ with $q\sigma$.\\
- Realize an automata $A_{x_k}$ for each variable $x_k$, with marked "states" $X_{m,k}$ and with as many events as elements in $G$ .\\
- Synchronize all automata.

\subsection{Synchronous composition of EFA's with common variables}

The synchronous composition of two EFA's $A=\langle Q^A, \Sigma^A,
\delta^A, q^A_i, Q^A_m, X^{A}, G^A, X^A_m \rangle$ and $B=\langle
Q^B, \Sigma^B, \delta^B, q^B_i, Q^B_m, X^{B}, G^B, X^B_m \rangle$
with common variables exists if and only if the automata are
consistent. The consistency condition requires that the action
functions triggered by a common enabled event $\sigma\in
\Sigma_{A}\cap \Sigma_{B}$ in the two automata update common
variables to the same values, i.e. if the $i^{ith}$ variable in
$x^{A}$ and the $j^{ith}$ variable in $x^{B}$ are the same
variable and $g_{q^A \sigma}(x^{A})\wedge g_{q^B \sigma}(x^{B})=1$
where $(q^A,q^B)\in Q^A\times Q^B$ then $(a_{q^A \sigma}(x^{A}))_i
= (a_{q^B \sigma}(x^{B}))_j$.

The consistency condition can be verified by inspection. However,
if the vectors $x^{A}$ or $x^{B}$ disable the corresponding guard
formula ($g_{q^A \sigma}(x^{A})\wedge g_{q^B \sigma}(x^{B})$) it
may happen that $a_{q^A \sigma}(x^{A}))_i \neq (a_{q^B
\sigma}(x^{B}))_j$ even though the two automata are consistent.
Therefore it may be a time-consuming task to verify the
consistency condition by inspection. In section
\ref{Implementation} we will present algorithms that includes a
consistency check of the synchronous
composition of two EFA's. \\

Formally we define the synchronous product $A||B$ of two EFA's
$A=\langle Q^A, \Sigma^A, \delta^A, q^A_i, Q^A_m, X^{A}, G^A,X^A_m
\rangle$ and $B=\langle Q^B, \Sigma^B, \delta^B, q^B_i, Q^B_m,
X^{B}, G^B,X^B_m
 \rangle$ as
\begin{eqnarray}
A||B=\langle Q^A\times Q^B, \Sigma^A \cup \Sigma^B , \delta, q_i,
Q^A_m\times Q^B_m, Z, G, Z_m \rangle.
\end{eqnarray}
To define the integer variables, the guard functions, the action
functions and the transition function for the product we make use
of the corresponding elements in $A$ and $B$.

 Let $z^{AB}$ denote the common variables of $x^{A}$ and $x^{B}$, and let $z^A$
and $z^B$ be the distinct variables of $x^{A}$ and $x^{B}$,
respectively. We define the integer variables $z$ of the
synchronous product of $A$ and $B$ as the vector $z=(z^A, z^{AB},
z^B)$. The guard functions $g_{q \sigma}(z)$ are defined as
\begin{eqnarray}\label{guard_sync_2}
g_{q \sigma}(z)=\left\{
\begin{array}{ll}
g_{q^A\sigma}(x^{A})\wedge g_{q^B\sigma}(x^{B}) \textrm{ if }\sigma\in \Sigma^A \cap \Sigma^B\\
g_{q^A\sigma}(x^{A}) \quad\quad \textrm{ if } \sigma\in \Sigma^A \setminus \Sigma^B\\
g_{q^B\sigma}(x^{B}) \quad\quad \textrm{ if }
\sigma\in\Sigma^B\setminus \Sigma^A
\end{array}\right.
\end{eqnarray}
If any one of the guards functions in the above definition are
undefined the guard function for the product $g_{q \sigma}$ is
also undefined. To simplify the notation further, we use "dummy"
guard functions in the following way. If $\sigma\in \Sigma^A
\setminus \Sigma^B$ we introduce fictional guards functions
$g_{q^B\sigma}(.)=1$ for $B$ and if $\sigma\in\Sigma^B\setminus
\Sigma^A$ we set $g_{q^A\sigma}(.)=1$. With this convention we can
write equation (\ref{guard_sync_2}) as
\begin{eqnarray}
g_{q \sigma}(z)= g_{q^A\sigma}(x^{A})\wedge g_{q^B\sigma}(x^{B}).
\end{eqnarray}
To define the action functions for $A||B$ we look at the
restrictions of the updating functions $a^A_\delta$ and
$a^B_\delta$ to the sets $Z^A$ and $Z^B$. The updating functions
are defined as (if $x^{A}$ and $x^{B}$ have no common variables
then $z=(x^{A},x^{B})$)
\begin{eqnarray}\label{action_sync_2}
a_{q \sigma}(z)=\left\{
\begin{array}{ll}
 \big(a^A_{q^A \sigma}(x^{A}), a^B_{q^B \sigma}(z^{B})\big)\quad\quad\sigma\in \Sigma^A \cap \Sigma^B\\
\big(a^A_{q^A \sigma}(x^{A}), z^{B}\big)\quad\quad\quad\sigma\in \Sigma^A \setminus \Sigma^B\\
\big(z^{A}, a^B_{q^B \sigma}(x^{B})\big)\quad\quad\quad\sigma\in
\Sigma^B \setminus \Sigma^A
\end{array}\right.
\end{eqnarray}
As in the definition of the guard functions the notion of being
undefined is inherited from the action functions of $A$ and $B$.
Similar to the "dummy" guard functions we introduce fictional
action functions. If $\sigma\in \Sigma^A \setminus \Sigma^B$ we
introduce fictional action functions $a_{q^B\sigma}(x^B)=x^B$ for
$B$ and if $\sigma\in\Sigma^B\setminus \Sigma^A$ we set
$a_{q^A\sigma}(x^A)=x^A$. If $x^{A}$ and $x^{B}$ have no common
variables then equation (\ref{action_sync_2}), that defines the
action functions of the product, simplifies to
\begin{eqnarray}
a_{q \sigma}(z)=
 \big(a^A_{q^A \sigma}(x^{A}), a^B_{q^B \sigma}(x^{B})\big).
\end{eqnarray}


 The transition function
$\delta(q,\sigma,z)$ is defined from $\delta^A$ and $\delta^B$ as
\begin{eqnarray}
\delta(q,\sigma,z)=\left\{
\begin{array}{ll}
\big(\delta^A(q^A,\sigma),\delta^B(q^B,\sigma),a_{q \sigma}(z)\big) \quad\quad \sigma\in \Sigma^A \cap \Sigma^B \quad g_{q \sigma}(z)=1\\
\big(\delta^A(q^A,\sigma),q^B,a_{q \sigma}(z)\big) \quad\quad\quad\quad\quad \sigma\in \Sigma^A \setminus \Sigma^B \quad g_{q \sigma}(z)=1\\
\big(q^A,\delta^B(q^B,\sigma),a_{q \sigma}(z)\big) \quad\quad\quad\quad\quad \sigma\in \Sigma^B \setminus \Sigma^A \quad g_{q \sigma}(z)=1\\
\textrm{
undefined}\quad\quad\quad\quad\quad\quad\quad\quad\quad\quad\quad\quad\quad\quad\quad\quad\quad
g_{q\sigma}(z)=0
\end{array}\right.
\end{eqnarray}

\section{Implementation of EFA's with hierarchy in Supremica}

\subsection{Supremica}

\subsection{Implementation}\label{Implementation}
\subsection{Examples}
\end{document}
