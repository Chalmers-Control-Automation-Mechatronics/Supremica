\documentclass{article}

\usepackage{amsthm,amsmath,amssymb, mathrsfs, graphicx,makeidx,multicol}

\newtheorem{theorem}{Theorem}


\begin{document}
\title{Modular Verification Using Extended Finite Automata}
\author{Markus Sk\"oldstam, Martin Byr\"od}
%\institute{Signals and Systems}
\maketitle
\tableofcontents
\section{Introduction}
To formally verify and validate PLC programs one needs a model of the controller.
There exists various methods to build models of
controllers,
\begin{enumerate}
   \item [1] modelling controllers with Signal Interpreted Petri Nets,
   \item [2] modelling controllers with SFC,
   \item [3] building transition systems from existing controllers,
   \item [4] using SOL and Function Blocks to describe controllers,
   \item [5] applying algorithms based on modular finite state machines.

   \end{enumerate}
 These models are usually translated to finite state machines
 or finite automata and validated with different supporting tools. The main difference
 between them lies in the formalism and the way the behavior of the plant is included.
 What one would like is a decision table that assists a user in finding
 the best approach for the application at hand. Here, we present a general framework using
 extended finite automata that models the plant and the specification.
 The plant and the specification are then synthesized, resulting in a model of the controller:)



\section{The formal definition}
An Extended Finite Automaton (EFA) is an
augmentation of the regular automaton with guard and action
formulas associated to the transitions. The transitions in the EFA
are enabled if and only if a guard formula is true. Moreover, when
a transition in an EFA is taken, updating actions of a set of
variables may follow. The formal definition of an EFA is here
explained by extending the definition of the regular finite automaton.
Typically, system behaviours are modelled using deterministic finite-state automata.
Nondeterminism is used here to support the possibility of having different guards on transitions
that are triggered by common events. To avoid contradictions when synchronizing
automata, we do not allow this for the updating actions of variables
that are shared between automata.\\

The regular finite nondeterministic automaton
$A$ is a $5$-tuple
\begin{eqnarray}
A=\langle Q,\Sigma,\delta, q_i,Q_m \rangle\nonumber
\end{eqnarray} where $Q$ is a finite set of states; $\Sigma$ (the alphabet) is a nonempty
set of events; $\delta$ is the transition relation mapping elements of $Q\times\Sigma$ into subsets of $Q$; $q_i$ is the
initial state and $Q_m \subseteq Q$ is the set of marked states.




First, we add to $A$ a set $X=X_1\times ... \times X_n$, where
$X_k$, $k=1,...,n$, are finite nonempty sets. For instance, $X_k$
could be any finite subset of $\mathbb{Z}$ or the set of states in  some
other finite deterministic automaton. The finite set $X$ is the
domain of definition of an $n$-tuple of variables
$x=(x_1,...,x_n)$ with initial value $x_0=(x_{10},...,x_{n0})$. If
a variable corresponds to the set of states in an automaton, we
shall call it a state variable. The initial value of a state
variable must equal the initial state of its automaton and we do
not admit updating actions of a state variable. To tell apart the
ordinary variables and the state variables we use the subscripts
$v$ and $s$, respectively. Using this convention we divide the
domain of definition and the variables into parts, $X=X_v \times
X_s$ and $x=(x_v,x_s)$.

 Let $T=\{(q,\sigma,p)\in Q \times \Sigma\times Q:
\delta(q,\sigma)=p\}$ denote the set of transitions in $A$ and let
$\tau_k$, $k=1,...,|T|$ denote the elements (transitions) of $T$. In the next step we
 add a $|T|$-tuple of function pairs $G=\big((g_{\tau_1},a_{\tau_1}),...,
(g_{\tau_{|T|}},a_{\tau_{|T|}})\big)$, where each pair
$(g_{\tau_k},a_{\tau_k})$ is associated to the corresponding
transition $\tau_k$ in $T$. $g_{\tau_k}$ is an evaluation of a
boolean formula (a guard formula) over $X$ i.e. $g_{\tau_k}:X
\rightarrow \{0,1\}$ saying if the corresponding transition
$\tau_k$ is enabled or not. $a_{\tau_k}$ is an updating (action)
function from $X_v$ to $X_v$ of the ordinary variables $x_v$. To
indicate the source($q$), event($\sigma$) and target($p$) of a
function pair we write $(g_{q\sigma p},a_{q\sigma p})$. The guards and
action formulas are partitioned into subsets $(g_{q\sigma},a_{q\sigma})=\{(g_{q\sigma p},a_{q\sigma p})\in G:
\delta(q,\sigma)=p\}$ whose elements have the same source and target.

Using the above notations we define an EFA as the
$9$-tuple $A_{e}=\langle Q, \Sigma, \delta_{e}, q_i, Q_m, X, G
,x_0, X_m \rangle$, where $X_m$ are the marked values of the
variables $x$ and $\delta_{e}: Q \times \Sigma \times X
\rightarrow Q\times X $ is a new extended version of the
transition mapping $\delta:Q \times \Sigma \rightarrow Q$. The
extended mapping $\delta_e$ is defined by means of
$\delta$, the subsets $(g_{q\sigma},a_{q\sigma})$ and the variables $x=(x_v, x_s)$ as
\begin{eqnarray}
\delta_{e}(q,\sigma,x)=\left\{
\begin{array}{ll}
(\delta(q,\sigma),a_{q \sigma}(x_v),x_s \big)\quad\quad g_{q \sigma}(x)=1\\
\textrm{undefined  } \quad\quad\quad\quad\quad\quad g_{q
\sigma}(x)=0
\end{array}\right..
\end{eqnarray}
The meaning of $\delta_{e}(q,\sigma,x)$ is that when scanning
the symbol $\sigma$ the extended automaton moves to anyone of the
states $p$ in the set $\delta(q,\sigma)$ whose guard $g_{q
\sigma p}(x)$ is true. When the transition takes place the variables are
updated by the action function.

The set of transitions in the extended automaton is
$T_e=\{(q,\sigma,x,p)\in Q \times \Sigma\times X\times Q:
\big(\delta(q,\sigma)=p \big)\wedge \big(g_{q
\sigma p}(x)=1\big)\}$. Notice that two transitions are
identical if and only if the events, the states and the values
of all variables are the same. Hence, a transition $(q,\sigma,p)$ in the
regular automaton $A$ is divided into as many transitions as the number
of elements in the set $\{x\in X:g_{q
\sigma p}(x)=1\}$. However, the
alphabets of the automata are the same and the language of the
extended automaton is subset of the language of the regular automaton:
\begin{eqnarray}
L(A_e)=\{s\in \Sigma^*:
\delta_e(q_i,s,x_0)!\}\subseteq \{s\in \Sigma^*:
\delta(q_i,s)!\}=L(A).
\end{eqnarray}
In some sense EFA's have small languages and a few number of
states to the expense of many transitions. This is an important observation
since it is the main reason why EFA's enable efficient modular algorithms to perform
verifications of supervisors.
\subsection{How to use the variables}
All logical systems can easily be represented by
EFA models. When using EFA's as a modelling tool it important to understand how to choose the variables.
A variable $x$ can represent many things, some examples  are: the states that an object can be in,
 the number of customers in a queue system, the different types of customers in a queue or the time it
takes for a process to execute different operations. In many applications the variables can be represented by
 integers or enumeration objects.

\subsubsection{Integer variables}

\subsubsection{Enumeration variables}


\section{Translating EFA to FA}

An EFA can be translated into its corresponding FA.
If $A$ is the translation of $A_e$ we shall use the notation
$A=trans(A_e)$. When an EFA is "flatten out" the number of states
increases to $Q \times X_v $. The reason for this is that all the
actions are removed and translated into events in the regular FA. Notice
that when $A_e$ only has ordinary variables $X=X_v$. If $A_e$ also has
state variables the guard conditions on these variables are evaluated. When a variable
corresponds to the set of states
in an other automata this is not possible. Therefore its guard conditions are set to
TRUE when the translation is performed. To simplify the
algorithm we shall restrict us to quite simple action- and guard
functions. We shall not compare variables with each other in the guard functions.
The updating of the variables are assumed to be independent
from each other. This implies that they can be written as
tuples of functions of one variable i.e.
$a_{q \sigma p}(x)=(a^1_{q \sigma p}(x),...,a^n_{q \sigma p}(x))=(a^1_{q
\sigma p}(x_1),...,a^n_{q \sigma p}(x_n))$. We shall assume that the
guards are parsed and written in disjunctive normal form $g_{q
\sigma p}(x)=g_1(x)\vee ...\vee g_j(x)= (g^{1}_{1}(x_1)\wedge
...\wedge g^{n}_{1}(x_n) )\vee ...\vee (g^{1}_{j}(x_1)\wedge
...\wedge g^{n}_{j}(x_n) )$.

These assumptions makes it
straight forward to translate an EFA to an FA. Given an extended
finite automaton $A_{e}=\langle Q, \Sigma, \delta, q_i, Q_m, X, G
,x_0, X_m \rangle$ with simple guard and
action functions, the steps are:\\



\begin{enumerate}
   \item [0] In the guard formulas (written in disjunctive normal form) the guards on
             the state variables are evaluated. If an and-clause belonging to a state variable is
             FALSE the "OR" component that it belongs to is removed, and if it is TRUE the and-clause
             is removed. The number of "OR" components in the modified
             guards will in what follows be denoted by $|g_{q \sigma p}|$.

   \item [1] Form a regular FA, $A'$, from $A_{e}$ by:
   \begin{enumerate}
   \item [(i)] removing $G$, $X$, $X_m$, $x_0$ and the alphabet $\Sigma$,

   \item [(ii)] form the alphabet of $A'$ by giving unique names to all events that are enabled in $A_{e}$,

   \item [(iii)] update the transitions by changing the names of the events accordingly.

   \item [(iv)] divide each event in $A'$ into $|g_{q \sigma p}|$ events with unique
   names,
   \item [(v)] split the corresponding transition into $|g_{q \sigma p}|$
transitions using the new event names.

   \end{enumerate}


 \item [2]  Realize an automaton $A_{k}$ for each one of the ordinary variables $x_{v}$,
 with as many events the total number of events in $A'$. The events in $A_k$ are associated
 to the guard function belonging
 to the right "OR" component of the guard. The transitions
correspond to the action functions and they are enabled if the corresponding
guard function is true.
 \item [3]  The wanted finite automaton $A$ is formed by synchronizing $A_{k}$, $k=1,...,m$ with $A'$.
 Here, the index $m$ is the total number of ordinary variables in $A_e$.
 \item [4] Change the event names of $A$ to the original ones. Add to $A$ the possible events
 in $A_{e}$ not present in any transitions, so that the two alphabets agree.
\end{enumerate}
The number of states in the variable automata equals the range of the corresponding variable.
In particular, the marked states of a variable automaton are those that correspond to the set
of marked values of that variable.

In the last two steps  of the algorithm we obtain a monolithic description of $A_e$. To avoid the so called
 state-space explosion problem, we must investigate the modular automata model obtained in step 2. However,
 the modular structure in the algorithm does not have the correct alphabet and can therefore not be used to
 verify specifications against a given plant model. For instance, suppose we have a model of the plant and
 a specification in EFA form. To analyze the specification against the plant and at the same time
use the modular structure in step 2 in the translation algorithm we must translate the
specification and the plant simultaneously.

Before we calculate a supervisor from the modular model obtained in step 2 we divide the alphabet of $A'$ (the
relabelled alphabet)
into as many subsets as the number of events (events that occur in transitions) in $A_e$. Each subset
$\Sigma_\sigma=\{\text{events in } A' \text{ obtained by relabelling } \sigma \}$
is an equivalence class representing the event $\sigma$ in the original alphabet. The supervisor
calculated from the modular model with the relabelled alphabet have one important restriction,
 it must either accept all events in an equivalence class or forbid all events in an equivalence class. Of course
  we need only to consider the controllable events in $A_e$.
 The key is to treat all the relabelled events as uncontrollable and control them
 simultaneously with the original controllable event
 of $A_e$ that defines the their equivalence class. This
 can easily be obtained using regular algorithms by adding a simple automata shown in fig.



Note that if two EFA's has common variables, we cannot first translate them
separately into standard FA and then synchronize them. Either this
has to be done simultaneously or the automata must be synchronized
on a higher level. The same problem occurs when translating EFA's with
state variables. However, by synchronizing all automata with
common variables an EFA with local ordinary variables can be constructed.



\section{The synchronous product}

The synchronous composition of two EFA's must consider that the automata
may have both events and variables in common. When synchronizing two automata
$A_e=\langle Q^A, \Sigma^A, \delta^A_e, q^A_i, Q^A_m,
X^A, G^{A}, x^{A}_0, X^{A}_m \rangle$ and $B_e=\langle Q^B, \Sigma^B,
\delta^B_e, q^B_i, Q^B_m, X^B, G^{B}, x^{B}_0, X^{B}_m \rangle$ that have
common variables one must check that $A_e$ and $B_e$ are
consistent. First of all, the common variables must have the same
initial values. Secondly, we require that the action functions
triggered by a common enabled event $\sigma\in \Sigma_{A}\cap
\Sigma_{B}$ in the two automata update common variables to the
same values, i.e. if the $i^{ith}$ variable in $x^{A}$ and the
$j^{ith}$ variable in $x^{B}$ are the same then $(a_{q^A \sigma}(x^{A}))_i = (a_{q^B
\sigma}(x^{B}))_j$. However, if the corresponding guard
formula is false it may happen that $a_{q^A \sigma p^A}(x^{A}))_i \neq (a_{q^B
\sigma p^B}(x^{B}))_j$ even though the two automata are consistent.
To avoid such situations we require that action functions of common
variables are identical?

Before we define the synchronous product we introduce two more
subscripts, $l$ and $c$. These are to be used for the variables.
The subscript $l$ indicates local variables and the subscript $c$
indicates common variables. In particular the subscripts can be
used together with the subscripts for ordinary variables and state
variables ($v$  and $s$, respectively). For instance, if we write
$x_{cv}$, then $x_{cv}$ is a common ordinary variable.

\subsection{Defining the product}\label{defining the product}

Formally we can write the synchronous product $A_e \| B_e$ of two
EFA's $A_e=\langle Q^A, \Sigma^A, \delta^A_e, q^A_i, Q^A_m, X^A,
G^{A}, x^{A}_0, X^{A}_m \rangle$ and $B_e=\langle Q^B, \Sigma^B,
\delta^B_e, q^B_i, Q^B_m, X^B, G^{B}, x^{B}_0, X^{B}_m \rangle$ as
\begin{eqnarray}
A_e \| B_e=\langle Q^A\times Q^B, \Sigma^A \cup \Sigma^B ,\nonumber\\
\delta^{A\|B}_e, (q^{A}_i, q^{B}_i),\nonumber\\
 Q^A_m\times Q^B_m, X^{A\|B},\nonumber\\
G^{A\|B}, x^{A\|B}_0, X^{A\|B}_m\rangle.
\end{eqnarray}
We need to define the variables, the guard functions, the action
functions and the transition function of the product. Let $x_{c}$
denote the common variables of $x^{A}$ and $x^{B}$, and let
$x^A_l$ and $x^B_l$ be the distinct variables of $x^{A}$ and
$x^{B}$, respectively. The variables of the product of $A$ and $B$
can be written as the vector $x^{A\|B}=(x^A_l, x_{c}, x^B_l)$. We
define the guard functions $g_{q \sigma}(x^{A\|B})$ as
\begin{eqnarray}\label{guard_sync_2}
g_{q \sigma}(x^{A\|B})=\left\{
\begin{array}{ll}
g_{q^A\sigma}(x^{A})\wedge g_{q^B\sigma}(x^{B}) \textrm{ if }\sigma\in \Sigma^A \cap \Sigma^B\\
g_{q^A\sigma}(x^{A}) \quad\quad \textrm{ if } \sigma\in \Sigma^A \setminus \Sigma^B\\
g_{q^B\sigma}(x^{B}) \quad\quad \textrm{ if }
\sigma\in\Sigma^B\setminus \Sigma^A
\end{array}\right.
\end{eqnarray}
If any one of the guards functions in the above definition are
undefined the guard function for the product $g_{q \sigma}$ is
also undefined. To simplify the notation further, we use "dummy"
guard functions in the following way. If $\sigma\in \Sigma^A
\setminus \Sigma^B$ we introduce fictional guards functions
$g_{q^B\sigma}(.)=1$ for $B$ and if $\sigma\in\Sigma^B\setminus
\Sigma^A$ we set $g_{q^A\sigma}(.)=1$. With this convention we can
write equation (\ref{guard_sync_2}) as
\begin{eqnarray}
g_{q \sigma}(x^{A\|B})= g_{q^A\sigma}(x^{A})\wedge
g_{q^B\sigma}(x^{B}).
\end{eqnarray}
By $a^B_{q \sigma}(x^{B}_{vl})$, we mean the restriction of
$a^B_{q^B \sigma}(x^{B}_{v})$ to the local ordinary variables of
$B$. The updating functions are defined as (if $x^{A}$ and $x^{B}$
have no common variables then $x^{A\|B}_v=(x^{A}_v,x^{B}_v)$)
\begin{eqnarray}\label{action_sync_2}
a_{q \sigma}(x^{A\|B}_v)=\left\{
\begin{array}{ll}
 \big(a^A_{q^A \sigma}(x^{A}_v), a^B_{q^B \sigma}(x^{B}_{vl})\big)\quad\quad\sigma\in \Sigma^A \cap \Sigma^B\\
\big(a^A_{q^A \sigma}(x^{A}_v), x^{B}_{vl}\big)\quad\quad\quad\quad\sigma\in \Sigma^A \setminus \Sigma^B\\
\big(x^{A}_{vl}, a^B_{q^B
\sigma}(x^{B}_v)\big)\quad\quad\quad\quad\sigma\in \Sigma^B
\setminus \Sigma^A
\end{array}\right.
\end{eqnarray}
As in the definition of the guard functions the notion of being
undefined is inherited from the action functions of $A$ and $B$.
Similar to the "dummy" guard functions we introduce fictional
action functions. If $\sigma\in \Sigma^A \setminus \Sigma^B$ we
introduce fictional action functions $a_{q^B\sigma}(x^B_v)=x^B_v$
for $B$ and if $\sigma\in\Sigma^B\setminus \Sigma^A$ we set
$a_{q^A\sigma}(x^A_v)=x^A_v$. If $x^{A}_v$ and $x^{B}_v$ have no
common variables then equation (\ref{action_sync_2}), that defines
the action functions of the product, simplifies to
\begin{eqnarray}
a_{q \sigma}(x^{A\|B}_v)=
 \big(a^A_{q^A \sigma}(x^{A}_v), a^B_{q^B \sigma}(x^{B}_v)\big).
\end{eqnarray}


 The transition function
$\delta^{A\|B}_e$ is defined using the regular transitions
functions $\delta^A$ and $\delta^B$ together with the guard and
action functions of the product,
\begin{eqnarray}
\delta^{A\|B}_{e}(q,\sigma,x^{A\|B})=\left\{
\begin{array}{ll}
\big(\delta^A(q^A,\sigma),\delta^B(q^B,\sigma),a_{q \sigma}(x^{A\|B}_v), x^{A\|B}_s \big) \quad\quad \sigma\in \Sigma^A \cap \Sigma^B \quad g_{q \sigma}(x^{A\|B})=1\\
\big(\delta^A(q^A,\sigma),q^B,a_{q \sigma}(x^{A\|B}_v), x^{A\|B}_s\big) \quad\quad\quad\quad\quad \sigma\in \Sigma^A \setminus \Sigma^B \quad g_{q \sigma}(x^{A\|B})=1\\
\big(q^A,\delta^B(q^B,\sigma),a_{q \sigma}(x^{A\|B}_v), x^{A\|B}_s\big) \quad\quad\quad\quad\quad \sigma\in \Sigma^B \setminus \Sigma^A \quad g_{q \sigma}(x^{A\|B})=1\\
\textrm{
undefined}\quad\quad\quad\quad\quad\quad\quad\quad\quad\quad\quad\quad\quad\quad\quad\quad\quad\quad\quad\quad\quad
g_{q\sigma}(x^{A\|B})=0
\end{array}\right.
\end{eqnarray}




\subsection{Properties of the product}



\subsubsection{When the product commutes with the translation algorithm}

The synchronous product $A_e\|B_e$ of two EFA that only have local ordinary
variables can be translated to $A\|B$ by translating
$A_e$ and $B_e$ separately into finite automata and then
synchronizing. In other words, the operations translating into
$FA$ and product commute, $trans({A\|B}_e) =trans(A_e)\|trans(B_e)$.
Before we prove this important property we shall look at the formulas
 for the product when the components have local ordinary variables.
 They become much easier to read.

In section \ref{defining the product} we defined the synchronous product $A_e\|B_e$
of two EFA's. Since here, we only consider automata with ordinary local variables,
the variables for the product simplifies to $x=(x^A,x^B)$. Hence, can write the product as
\begin{eqnarray}
A_e\|B_e&=&\langle Q^A\times Q^B, \Sigma^A \cup \Sigma^B ,
\delta^{A\|B}_e, (q^A_i,q^B_i) ,\nonumber\\
&& Q^A_m\times Q^B_m, X^A \times X^B, G^{A\|B},(x^A_0,x^B_0),
X^A_m \times X^B_m \rangle.
\end{eqnarray}
Removing unnecessary  superscripts we can write $A_e\|B_e=\langle
Q, \Sigma, \delta_e, q_i, Q_m, X, G ,x_0, X_m \rangle$. To simplify the
notation and increase readability, we use "dummy" guard functions
and fictional action function as in section \ref{defining the product}.
Doing this, we can write the guard and action formulas for the product as
\begin{eqnarray}
g_{q \sigma}(x)= g_{q^A\sigma}(x^{A})\wedge g_{q^B\sigma}(x^{B})
\end{eqnarray}
and
\begin{eqnarray}
a_{q \sigma}(x)=
 \big(a_{q^A \sigma}(x^{A}), a_{q^B \sigma}(x^{B})\big).
\end{eqnarray}
The transition function $\delta^{A\|B}_e$ simplifies since we do not have any
state variables ($\delta^A$ and $\delta^B$ are the transition functions
for the corresponding regular automata)
\begin{eqnarray}
\delta^{A\|B}_e(q,\sigma,x)=\left\{
\begin{array}{ll}
\big(\delta^A(q^A,\sigma),\delta^B(q^B,\sigma),a_{q \sigma}(x)\big) \quad\quad \sigma\in \Sigma^A \cap \Sigma^B \quad g_{q \sigma}(x)=1\\
\big(\delta^A(q^A,\sigma),q^B,a_{q \sigma}(x)\big) \quad\quad\quad\quad\quad \sigma\in \Sigma^A \setminus \Sigma^B \quad g_{q \sigma}(x)=1\\
\big(q^A,\delta^B(q^B,\sigma),a_{q \sigma}(x)\big) \quad\quad\quad\quad\quad \sigma\in \Sigma^B \setminus \Sigma^A \quad g_{q \sigma}(x)=1\\
\textrm{
undefined}\quad\quad\quad\quad\quad\quad\quad\quad\quad\quad\quad\quad\quad\quad\quad\quad\quad
g_{q\sigma}(x)=0
\end{array}\right.
\end{eqnarray}


\begin{theorem}
Two automata with no common variables $A_e$ and $B_e$ fulfills
$trans({A\|B}_e) =trans(A_e)\|trans(B_e)$.
\end{theorem}

\subsubsection{Associativity}

Consider three EFA's $A$, $B$ and $C$. From the definition of the
synchronization we shall now show that
 $(A\|B)\|C=A\|(B\|C)$ i.e. we have:
\begin{theorem}
   Full synchronous composition of EFA's is an associative
   operation.
\end{theorem}

\noindent \textbf{Proof}: We need to prove that $(A\|B)\|C =
A\|(B\|C)$. For simplicity, we assume that the three automata have
no common variables.
\newline\newline
\noindent The 9-tuple representing $(A\|B)\|C$ is
\begin{center}
\begin{tabular}{ll}
  $(A\|B)\|C =$ & $\Big \langle (Q^{A} \times Q^{B}) \times Q^{C},$\\
  & $ \big(\Sigma^{A} \bigcup \Sigma^{B} \big) \bigcup \Sigma^{C},$\\
  & $\delta^{(A\|B)\|C},$\\
  & $\big((q^{A}_i, q^{B}_i) , q^{C}_i\big),$\\
  & $(Q^{A}_m \times Q^{B}_m) \times Q^{C}_m,$\\
  & $\big[[X^{A}, X^{B}], X^{C}]$\\
  & $G^{(A\|B)\|C},$\\
  & $\big((x^{A}_0, x^{B}_0) , x^{C}_0\big),$\\
  & $\big[[X^{A}_m, X^{B}_m], X^{C}_m \big] \Big \rangle$
\end{tabular}
\end{center}

\noindent In order to establish associativity we now have to show
this for each of the elements in the tuple. For the states, the
alphabet, the initial state, the marked states and the variables,
associativity is clear by the laws of set operations. We will now
in turn deal with the guard/action functions and the transition
function.

For the guard functions and the action functions of the composite
automaton we have
\begin{eqnarray}
g^{(A\|B)\|C}_{q \sigma}(x)&=& g_{q^{AB}\sigma}(x^{AB})\wedge
g_{q^C\sigma}(x^{C})\nonumber\\
&=&g_{q^A\sigma}(x^{A})\wedge g_{q^B\sigma}(x^{B})\wedge
g_{q^C\sigma}(x^{C})\nonumber\\
&=& g_{q^{A}\sigma}(x^{A})\wedge g_{q^{BC}\sigma}(x^{BC})\nonumber\\
&=& g^{A\|(B\|C)}_{q \sigma}(x)
\end{eqnarray}
and
\begin{eqnarray}
a^{(A\|B)\|C}_{q \sigma}(x)&=&
 \big(a_{q^{AB} \sigma}(x^{AB}), a_{q^{C}
 \sigma}(x^{C})\big)\nonumber\\
&=&\big(a_{q^{A} \sigma}(x^{A}), a_{q^{B}
 \sigma}(x^{B}), a_{q^{C}
 \sigma}(x^{C})\big)\nonumber\\
&=&\big(a_{q^{A} \sigma}(x^{A}), a_{q^{BC}
 \sigma}(x^{BC})\big)\nonumber\\
&=&a^{A\|(B\|C)}_{q \sigma}(x).
\end{eqnarray}
This shows that $G^{(A\|B)\|C}=G^{A\|(B\|C)}$, and since the
transition function for the standard $FA$ is associative it
follows that $\delta^{(A\|B)\|C}=\delta^{A\|(B\|C)}$.
\begin{flushright}$\square$\end{flushright}




\section{Translating hierarchical structures to EFA's} It is well
known that hierarchical structures of states and transitions
reduce the descriptive complexity of automata models. We shall in
this section show how the hierarchy in state charts can be
translated into EFA's.

In what follows, states are represented by boxes and hierarchy is
illustrated by being inside or outside the boxes. Arrows between
states represents transitions and are labelled with the lower case
Greek letters, $\alpha$-$\omega$. An arrow that leaves a box
applies to all sub-states inside the box. The common way to enter
a group of states its by the system's history in that group. The
simplest way of doing this is to enter the most recently visited
state in the group. Hierarchical structures with quite general
history functions can easily be translated into EFA's.

\section{Remarks}
\begin{enumerate}
   \item [1] Shall we remove marked integer values from our definition of
EFA's? Will the user only want to mark the states?

    \item [2] Waters can only load one file. This must be fixed
    before synch. of EFA's.

    \item [3] Can we translate EFA's to Tord's stuff?

    \item [4] We must find examples where EFA's are easy to use.

    \item [5] We want to find examples where it is better to
    synch. two EFA's than to translate to FA's and then synch.

    \item [6] Find smart algorithm's  than convert FA's to EFA's.

     \item [7] Decide what to write in our article.

    \item [8] Look at EFA's as a supervisory control tool? this is
    done in all standard papers.

    \end{enumerate}


\section{Implementation}\label{Implementation}
\subsection{Examples}

\begin{thebibliography}{99}

\bibitem{Chen} Y-L. Chen and F. Lin (2000). Modeling of Discrete Event Systems
Using Finite State Machines with Parameters, emph{Proc. of the 2000 IEEE International Conference on Control Applications},
Anchorag, Alaska, USA, pp. 941-946.

\bibitem{Yang}Y. Yang and P. Gohari(2005). Embedded supervisory control of discrete-event systems,
\emph{Proc. of the 2005 IEEE International Conference on Automation Science and Engineering},
Edmonton, Canada, pp. 410-415.


\end{thebibliography}

\end{document}
