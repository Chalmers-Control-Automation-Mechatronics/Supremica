\documentclass{article}


\usepackage{amsthm,amsmath,amssymb}

\newtheorem{theorem}{Theorem}

\begin{document}
\tableofcontents
\section{Introduction}

\subsection{Background}

\section{Extended Finite Automaton (EFA)}
An Extended Deterministic Finite Automaton (EFA) is an
augmentation of a regular finite automaton. The regular automaton
$A$ is defined as $A=<Q,\Sigma,\delta, q_i,Q_m>$. Let
$T=\{(q,\sigma)\in Q \times \Sigma: \delta(q,\sigma)!\}$ denote
the set of transitions in the deterministic automaton $A$ and let
$\tau_k$, $k=1,...,|T|$ denote the elements (transitions) of $T$.

The automaton $A$ is extended by a set $X=(x_1,...,x_n)$ of $n$
integer variables $x_{k}$ and by a $|T|$-tuple of function pairs
$G=\big((g_{\tau_1},a_{\tau_1}),...,
(g_{\tau_{|T|}},a_{\tau_{|T|}})\big)$, where each pair
$(g_{\tau_k},a_{\tau_k})$ is associated to the corresponding
transition $\tau_k$ in $T$. $g_{\tau_k}$ is an evaluation of a
boolean formula (a guard formula) over $X$ i.e.
$g_{\tau_k}:{\mathbb{N}}^{n}\rightarrow \{0,1\}$ saying if the
corresponding transition $\tau_k$ is enabled or not. $a_{\tau_k}$
is an updating (action) function from ${\mathbb{N}}^{n}$ to
${\mathbb{N}}^{n}$ of the integer variables $X$.

With the above notations we will formally define an EFA as the
$8$-tuple $A_{e}=<Q, \Sigma, \delta_{e}, q_i, Q_m, X, G>$, where
$\delta_{e}: Q \times \Sigma \times X \rightarrow Q\times X $ is
an extention of the transition function $\delta:Q \times \Sigma
\rightarrow Q$. $\delta_e$ is defined by means of $\delta$ and the
elements of $G$ as
\begin{eqnarray}
\delta_{e}(q,\sigma,X)=\left\{
\begin{array}{ll}
(\delta(q,\sigma),a_{q \sigma}(X)\big)\quad\quad g_{q \sigma}(X)=1\\
\big(q,X \big)\quad\quad\quad\quad g_{q \sigma}(X)=0
\end{array}\right.
\end{eqnarray}
If the guard formula $g_{q \sigma}$ is undefined then
$\delta_{e}(q,\sigma,X)$ is undefined.



\subsection{Synchronous composition of EFA's}
The synchronous composition of two EFA's $A=<Q^A, \Sigma^A,
\delta^A, q^A_i, Q^A_m, X, G^A>$ and $B=<Q^B, \Sigma^B, \delta^B,
q^B_i, Q^B_m, Y, G^B>$ exists if and only if the automata are
consistent. The consistency condition requires that the action
functions triggered by a common enabled  event $\sigma\in
\Sigma_{A}\cap \Sigma_{B}$ in the two automata update common
variables to the same values, i.e. if the $i^{ith}$ variable in
$X$ and the $j^{ith}$ variable in $Y$ are the same variable and
$g_{\alpha \sigma}(X)\wedge g_{\beta \sigma}(Y)=1$ where
$(\alpha,\beta)\in Q^A\times Q^B$ then $(a_{\alpha \sigma}(X))_i =
(a_{\beta \sigma}(Y))_j$.

The consistency condition can be verified by inspection. However,
if the vectors $X$ or $Y$ disable the corresponding guard formula
($g_{\alpha \sigma}(X)\wedge g_{\beta \sigma}(Y)$) it may happen
that $a_{\alpha \sigma}(X))_i \neq (a_{\beta \sigma}(Y))_j$ even
though the two automata are consistent. Therefore it may be a
time-consuming task to verify the consistency condition by
inspection. In section \ref{Implementation} we will present
algorithms that includes a consistency check of the synchronous
composition of two EFA's. \\

Formally we define the synchronous product $A||B$ of two EFA's
$A=<Q^A, \Sigma^A, \delta^A, q^A_i, Q^A_m, X, G^A>$ and $B=<Q^B,
\Sigma^B, \delta^B, q^B_i, Q^B_m, Y, G^B>$ as
\begin{eqnarray}
A||B=<Q^A\times Q^B, \Sigma^A \cup \Sigma^B , \delta, q_i,
Q^A_m\times Q^B_m, X \cup Y, G>.
\end{eqnarray}
To define the guard functions, the action functions and the
transition function for the product we make use of the
corresponding functions in $A$ and $B$.

The guard functions $g_{q \sigma}(Z)=g_{q \sigma}(X\cup Y)$ are
defined as
\begin{eqnarray}
g_{q \sigma}(Z)=\left\{
\begin{array}{ll}
g_{q^A\sigma}(X)\wedge g_{q^B\sigma}(Y) \quad\quad\sigma\in \Sigma^A \cap \Sigma^B\\
g_{q^A\sigma}(X) \quad\quad\sigma\in \Sigma^A \setminus \Sigma^B\\
g_{q^B\sigma}(Y) \quad\quad\sigma\in \Sigma^B \setminus \Sigma^A\\
\textrm{undefined       otherwise}
\end{array}\right.
\end{eqnarray}
If $X\cap Y\neq \emptyset$ we introduce $Y'=Y\setminus X$ and
$X'=X\setminus Y$ and look at the restrictions of the updating
functions $a^A_\delta$ and $a^B_\delta$ to the sets $X'$ and $Y'$.
The updating functions are defined as (if $X\cap Y=\emptyset$ then
$Y'=Y$ and $X'=X$)
\begin{eqnarray}
a_{q \sigma}(Z)=\left\{
\begin{array}{ll}
 \big(a^A_{q^A \sigma}(X), a^B_{q^B \sigma}(Y')\big)\quad\quad\sigma\in \Sigma^A \cap \Sigma^B\\
\big(a^A_{q^A \sigma}(X), Y'\big)\quad\quad\sigma\in \Sigma^A \setminus \Sigma^B\\
\big(X', a^B_{q^B \sigma}(Y)\big)\quad\quad\sigma\in \Sigma^B \setminus \Sigma^A\\
\textrm{undefined       otherwise}
\end{array}\right.
\end{eqnarray}


 The transition function
$\delta(q,\sigma,Z)=\delta(q^A,q^B,\sigma,X\cup Y)$ is defined
from $\delta^A$ and $\delta^B$ as
\begin{eqnarray}
\delta(q,\sigma,Z)=\left\{
\begin{array}{ll}
\big(\delta^A(q^A,\sigma),\delta^B(q^B,\sigma),a_{q \sigma}(Z)\big) \quad\quad \sigma\in \Sigma^A \cap \Sigma^B \quad g_{q \sigma}(Z)=1\\
\big(\delta^A(q^A,\sigma),q^B,a_{q \sigma}(Z)\big) \quad\quad \sigma\in \Sigma^A \setminus \Sigma^B \quad g_{q \sigma}(Z)=1\\
\big(q^A,\delta^B(q^B,\sigma),a_{q \sigma}(Z)\big) \quad\quad \sigma\in \Sigma^B \setminus \Sigma^A \quad g_{q \sigma}(Z)=1\\
\textrm{undefined       otherwise}
\end{array}\right.
\end{eqnarray}

\noindent By the above definition of synchronization we now obtain
the following property

\begin{theorem}
   Full synchronous composition of EFAs is an associative
   operation.
\end{theorem}

\noindent \textbf{Proof}: We show this by considering three
automata $A, A'$ and $A''$ and showing that $(A\|A')\|A'' =
A\|(A'\|A'')$.

The 8-tuple representing $(A\|A')\|A''$ is

\begin{center}
\begin{tabular}{ll}
  $(A\|A')\|A'' =$ & $\Big \langle (Q \times Q') \times Q'',$\\
  & $ \big(\Sigma \bigcup \Sigma' \big) \bigcup \Sigma'',$\\
  & $\delta^{(A\|A')\|A''},$\\
  & $\big((q_i, q'_i) , q''_i\big),$\\
  & $(Q_m \times Q_m') \times Q_m'',$\\
  & $\big(X \bigcup X') \bigcup X'',$\\
  & $G^{(A\|A')\|A''},$\\
  & $\big(X_m \bigcup X'_m) \bigcup X''_m \Big \rangle$
\end{tabular}
\end{center}

\noindent In order to establish associativity we now have to show
this for each of the elements in the 8-tuple. For the states, the
alphabet, the initial state, the marked states and the variables,
associativity is clear by the laws of set operations. We will now
in turn deal with the guard/action functions and the transition
function.

For the guard function $g^{(A\|A')\|A''} : X^c \rightarrow
\{\mathrm{true}, \mathrm{false}\}$ of the composite automaton we
have by definition

\begin{equation}
  g^{(A\|A')\|A''}(X^c) = \big[g(X) \wedge g'(X') \big] \wedge g''(X'') = g(X) \wedge \big[g'(X') \wedge g''(X'')] = g^{A\|(A'\|A'')}(X^c)
\end{equation}

\noindent Now consider the action function $a^{(A\|A')\|A''} : X^c
\rightarrow X^c$. Here things get a bit more messy, since we need
to make sure that each variable is updated at most one time even
though we might have $X\bigcap X'\bigcap X'' \neq \emptyset$

\section{Implementation of EFA's with hierarchy in Supremica}

\subsection{Supremica}

\subsection{Implementation}\label{Implementation}
\subsection{Examples}
\end{document}
