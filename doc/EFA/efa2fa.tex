\documentclass{article}

\usepackage{amsthm,amsmath,amssymb}

\newtheorem{theorem}{Theorem}

\pagestyle{empty}

\begin{document}

\section*{The translation of an EFA to  an FA}
The main body of theory and algorithms within supervisory control
theory has been developed for standard finite automata. For the
extended finite automata to be useful we need to fit them within
this framework. In other words, in order to do formal analysis
based on these new automata, what we need is a way to convert EFAs
into standard FAs.

Fortunately, as it turns out, there is a rather elegant
transformation of EFAs into FAs based on synchronization. In
essence, the recipe is as follows:

\begin{enumerate}
    \item Form a stripped EFA by removing the variables and all
    guard/action pairs from the EFA. This yields a regular FA.
    \item Introduce a new unique event for each guard/action pair and
     relabel the transitions of the stripped automaton correspondingly.
    \item Introduce \emph{one} automaton $A_{x_k}$ for each variable $x_k$ with
    the number of states equal to the range of $x_k$.

    \item
    For the states of the variable automata $A_{x_k}$ that fulfill the guard conditions,
    add transitions that realize the updating of the variables as specified by the action
    functions.

    \item Construct the FA-equivalent of the EFA by synchronizing
    the stripped EFA with all variable automata $A_{x_k}$.
    \item Change back to original event names.
\end{enumerate}


\noindent Below, an example of how this can be done is given.
\end{document}
